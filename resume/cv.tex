\documentclass{article}
\usepackage{hyperref}

\begin{document}

\title{Sergei Chudov \\ \large Senior Software Engineer }
\author{}
\date{}
\maketitle

\section*{Introduction}
\begin{quote}
I'm a software engineer with 5 years of experience. Now I am in Yerevan and ready for relocation or remote job.

Currently I am learning new technologies, upgrade my knowledge in software development and learning English.
\end{quote}

\section*{Contact Information}
\begin{itemize}
    \item Email address: \href{mailto:chudov42@gmail.com}{chudov42@gmail.com}
    \item LinkedIn: \href{https://www.linkedin.com/in/window-generator}{linkedin.com/in/window-generator}
    \item Github: \href{https://github.com/WindowGenerator}{github.com/WindowGenerator}
    \item Phone: +37491186658
    \item Telegram: \href{https://t.me/WidowGenerator}{@WindowGenerator}
\end{itemize}

\section*{Work Experience}
\subsection*{Senior Software Engineer}
\textit{Bostongene}, Yerevan, Armenia - (Jun 2022 - Present)
\begin{itemize}
    \item I participate in creating internal projects to optimize the work of internal customers. I am responsible for the stability of services running in Kubernetes.  
    \item Achievements:
    \begin{itemize}
        \item Saved one of the key databases by writing a script \href{https://github.com/WindowGenerator/netfs_unlker}{netfs\_unlker}. This action preserved 2 weeks of work for the teams working with these databases.
        \item Participated in gathering requirements and designing the architecture, as well as the subsequent development of this service to facilitate policy annotation billing.
        \item Implemented best practices in our team's development process, including tests, linters, and formatters, which helped increase our productivity.
        \item Deployed and maintained multiple production services in a Kubernetes environment, including Airflow, adapting to changing requirements and system loads.
        \item In the shortest possible time (4 days), I developed a production library with tests and CI/CD, which included a small language for converting patient parameters and their relationships into a machine-readable format.
    \end{itemize}
    \item Responsibilities:
    \begin{itemize}
        \item Moving bioinformatics algorithms to production
        \item Gathering client requirements and converting them into code
        \item Maintaining the functionality and monitoring services in Kubernetes
        \item Writing and supporting production Python packages
        \item Implementing frontend, backend, and pipelines in existing and new projects
        \item Writing Helm charts and setting up CI/CD (Concourse) for projects
    \end{itemize}
    \item Stack:
    \begin{itemize}
        \item Python (FastAPI, SQLAlchemy)
        \item Linux
        \item TypeScript (JavaScript, React, Redux)
        \item Docker, Kubernetes, Helm, Concourse, gitlab ci
        \item RabbitMQ, Redis, PostgreSQL, Neo4j, Apache Kafka
        \item Software Design, Software Infrastructure, Software Systems Engineering, Software Development
    \end{itemize}
\end{itemize}

\subsection*{Middle Software Engineer}
\textit{USSC Ldt.}, Yekaterinburg, Russia - (Oct 2019 - Mar 2022)
\begin{itemize}
    \item Participated in the development of a system for detecting and preventing information security incidents in industrial networks.
    \item Achievements:
    \begin{itemize}
        \item Simplified the implementation of user authorization, which reduced the number of bugs in this functionality.
        \item Rewrote the complex and frequently breaking master-master interaction functionality, and covered it with tests, significantly reducing bugs.
        \item Developed a backup service for main databases Mysql, Postgres.
        \item Implemented index rotation in Elasticsearch.
        \item Implemented a tool for generating TypeScript clients based on OpenAPI specifications. This reduced memory usage for these clients and improved the readability of method signatures on the client side.
    \end{itemize}
    \item Responsibilities:
    \begin{itemize}
        \item Designed microservices in python
        \item Optimized SQL in MySQL, PostgreSQL
        \item Developed and supported of authorization
        \item Wrote Docker images to build from source
        \item Wrote unit and integration tests
        \item Edited frontend when I didn't want to distract the frontend developers
    \end{itemize}
    \item Stack:
    \begin{itemize}
        \item Python (aiohttp, asyncio, FastAPI, Flask, SQLAlchemy)
        \item Linux
        \item Docker, Jenkins
        \item Elasticsearch, Mysql, Postgres, RabitMQ
        \item TypeScript (Angular, Javascript)
    \end{itemize}
\end{itemize}

\subsection*{Trainee ML Engineer}
\textit{NPO Avtomatiki}, Yekaterinburg, Russia - (Mar 2019 - May 2019)
\begin{itemize}
    \item Implemented prototype of system that allows you to identify people with high accuracy in real time, the data is created by a video camera.
    \item Responsibilities:
    \begin{itemize}
        \item Trained additional parts of YOLOV3 model
        \item Optimized of YOLOV3 performance on weak hardware
        \item Selected a ML model for object recognition
    \end{itemize}
    \item Stack:
    \begin{itemize}
        \item Python, tensorflow, keras, sklearn, Multiprocessing on Python
    \end{itemize}
\end{itemize}

\section*{Skills}
\subsection*{Development}
Python - FastAPI - aiohttp - Flask - Docker - Kubernetes - Concouse - Linux - asyncio - MySQL - PostgreSQL - Redis - RabbitMQ - Elasticsearch - Neo4j - Typescript - Angular - React - Kafka

\section*{Contributions}
\begin{itemize}
    \item \href{https://github.com/mosquito/aio-pika}{aio-pika}
    \item \href{https://github.com/astral-sh/ruff}{ruff}
    \item \href{https://github.com/langchain-ai/chat-langchain}{chat-langchain}
\end{itemize}

\section*{Pet-Projects}
\begin{itemize}
    \item \href{https://github.com/WindowGenerator/starlette-responses-kit}{starlette-responses-kit} - extension containing custom responses for starlette.
    \item \href{https://github.com/WindowGenerator/netfs_unlker}{netfs\_unlker} - netfs\_unlcker is a Rust library and command-line interface designed to provide unlock 'fcntl' shared locks in netapp's
\end{itemize}

\section*{Languages}
\subsection*{Russian}
Native speaker 
\subsection*{English}
B1 

\section*{Education}
\subsection*{Ural State Technical University}
\subsubsection*{(Bachelor of Science (Unfinished))}
Sep 2015 - Oct 2017
Radiophysicist

\subsubsection*{(Bachelor of Engineering)}
Oct 2017 - Jul 2020
Design and Technology of Electronic Facilities

\subsection*{Coursera}
Aug 2020 - Aug 2020
\href{https://www.coursera.org/learn/algorithms-part1}{Algorithms, Part I}

Aug 2021 - Sep 2021
\href{https://www.coursera.org/learn/vvedenie-mashinnoe-obuchenie}{Introduction to Machine Learning}

\end{document}